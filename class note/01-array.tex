% !TeX root = ./index.tex
\section{Array}
The javascript \textbf{Array} object is a global object that is used in the construction of arrays; which are high-level, list-like objects.
% \medskip
\begin{lstlisting}
const strings = ["a", "b", "c", "d"];
// 4 * 4 = 16 bytes of storage

strings[2]; 
// expected output: c
\end{lstlisting}
Every element of an array consumes 4 bytes of memory. If the array has four elements, it's consuming sixteen bytes of memory.

\subsection{Create an Array}
\begin{lstlisting}
var fruits = ['Apple', 'Banana'];
console.log(fruits.length);
// expected output: 2
\end{lstlisting}

\subsection{Methods in the array}

\subsubsection{Push}
The \textbf{push()} methods adds one or more elements to the end of an array and returns the new length of the array. The Big O notation for this method is O(1).
\begin{lstlisting}
const strings = ["a", "b", "c", "d"];

strings.push("e");

console.log(strings); 
// expected output: Array ["a", "b", "c", "d", "e"]
\end{lstlisting}


\subsubsection{Pop}
The \textbf{pop()} method removes the \textbf{last} element from an array and returns that element. This method changes the length of the array. The Big O notation for this method is O(1).

\begin{lstlisting}
const strings = ["a", "b", "c", "d"];

strings.pop();
// expected output: ["a", "b", "c"]
\end{lstlisting}

\subsubsection{Unshift}
The \textbf{unshift()} methods adds one or more elements to the beginning of an array and returns the new length of the array. The Big O notation for this method is O(n).

\begin{lstlisting}
const strings = ["a", "b", "c", "d"];

strings.unshift("x");
// expected output: ["x", "a", "b", "c", "d"]
\end{lstlisting}

\subsubsection{Splice}
The \textbf{splice()} method changes the contents of an array by removing existing elements and/or adding new elements. The Big O Notation for this method is O(n).

\begin{lstlisting}
const strings = ["a", "b", "c", "d"];

strings.splice(2, 0, "alien")
// expected output: ["a", "b", "c", "alien", "d"]
\end{lstlisting}

\textbf{Syntax}
\begin{lstlisting}
  array.splice(start[, deleteCount[, item1[, item2[, ...s]]]])
\end{lstlisting}

\newpage
